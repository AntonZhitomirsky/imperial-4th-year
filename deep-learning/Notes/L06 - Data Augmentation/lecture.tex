\documentclass[11pt]{article}

% some definitions for the title page
\newcommand{\reporttitle}{Data Augmentation}
\newcommand{\reportdescription}{Bare-bones summary of the lectures}

% load some definitions and default packages
%---------------------------------------------------------------------------
%	PACKAGES AND OTHER DOCUMENT CONFIGURATIONS
%---------------------------------------------------------------------------

\usepackage[twoside]{fancyhdr}
\usepackage{csquotes}

\usepackage[a4paper,hmargin=2.0cm,vmargin=1.0cm,includeheadfoot]{geometry}
% \usepackage{natbib} % for bibliography
\usepackage{biblatex}
\usepackage{tabularx,longtable,multirow,subfigure,caption}%hangcaption
\usepackage{fancyhdr} % page layout
\usepackage{url} % URLs
\usepackage[english]{babel}
\usepackage{graphicx}
\usepackage{rotating}
\usepackage{dsfont}
\usepackage{epstopdf} % automatically replace .eps with .pdf in graphics
% \usepackage{backref} % needed for citations
\usepackage{array}
\usepackage{latexsym}
\usepackage[pdftex,hypertexnames=false,colorlinks]{hyperref} % provide links in pdf (had pagebackref)
\usepackage{booktabs}
\usepackage{wrapfig}
\usepackage{caption}  % Required for \captionof
\usepackage{float} % for H option in figures
\usepackage{amssymb}
\usepackage{amsmath}
\usepackage{amsthm}
\usepackage{mathtools} % for 'dcases*' env.
\usepackage[nottoc]{tocbibind}

%%% Default fonts
\renewcommand*{\rmdefault}{bch}
\renewcommand*{\ttdefault}{cmtt}

%%% Default settings (page layout)
\setlength{\parindent}{0em}  % indentation of paragraph
\setlength{\parskip}{.3em}
\setlength{\itemsep}{0.mm}

\setlength{\headheight}{14.5pt}
\pagestyle{fancy}

\fancyfoot[ER,OL]{\thepage}%Page no. in the left on odd pages and on right on even pages

\fancyfoot[OC,EC]{\sffamily }
\renewcommand{\headrulewidth}{0.1pt}
\renewcommand{\footrulewidth}{0.1pt}
\captionsetup{margin=10pt,font=small,labelfont=bf}

% LISTINGS ammendments
\usepackage{listings}
\usepackage{color}

\definecolor{mygreen}{rgb}{0,0.6,0}
\definecolor{mygray}{rgb}{0.5,0.5,0.5}
\definecolor{mymauve}{rgb}{0.58,0,0.82}

\lstset{ 
  postbreak=\mbox{\textcolor{red}{$\hookrightarrow$}\space},
  backgroundcolor=\color{white},   % choose the background color; you must add \usepackage{color} or \usepackage{xcolor}; should come as last argument
  basicstyle=\footnotesize,        % the size of the fonts that are used for the code
  breakatwhitespace=false,         % sets if automatic breaks should only happen at whitespace
  breaklines=true,                 % sets automatic line breaking
  captionpos=b,                    % sets the caption-position to bottom
  commentstyle=\color{mygreen},    % comment style
%   deletekeywords={...},            % if you want to delete keywords from the given language
%   escapeinside={\%*}{*)},          % if you want to add LaTeX within your code
  extendedchars=true,              % lets you use non-ASCII characters; for 8-bits encodings only, does not work with UTF-8
  firstnumber=1,                % start line enumeration with line 1000
  frame=single,	                   % adds a frame around the code
  keepspaces=true,                 % keeps spaces in text, useful for keeping indentation of code (possibly needs columns=flexible)
  columns=fullflexible,
  keywordstyle=\color{blue},       % keyword style
  language=python,                 % the language of the code
  % morekeywords={*,...},            % if you want to add more keywords to the set
  numbers=left,                    % where to put the line-numbers; possible values are (none, left, right)
  numbersep=5pt,                   % how far the line-numbers are from the code
  numberstyle=\tiny\color{mygray}, % the style that is used for the line-numbers
  rulecolor=\color{black},         % if not set, the frame-color may be changed on line-breaks within not-black text (e.g. comments (green here))
  showspaces=false,                % show spaces everywhere adding particular underscores; it overrides 'showstringspaces'
  showstringspaces=false,          % underline spaces within strings only
  showtabs=false,                  % show tabs within strings adding particular underscores
  stepnumber=1,                    % the step between two line-numbers. If it's 1, each line will be numbered
  stringstyle=\color{mymauve},     % string literal style
  tabsize=2,	                   % sets default tabsize to 2 spaces
  title=\lstname% show the filename of files included with \lstinputlisting; also try caption instead of title
}

% Here, you can define your own macros. Some examples are given below.

\newcommand{\R}[0]{\mathds{R}} % real numbers
\newcommand{\Z}[0]{\mathds{Z}} % integers
\newcommand{\N}[0]{\mathds{N}} % natural numbers
\newcommand{\C}[0]{\mathds{C}} % complex numbers
\renewcommand{\vec}[1]{{\boldsymbol{{#1}}}} % vector
\newcommand{\mat}[1]{{\boldsymbol{{#1}}}} % matrix


\begin{document}

% Include the title page
\begin{titlepage}

    \newcommand{\HRule}{\rule{\linewidth}{0.5mm}} % Defines a new command for the horizontal lines, change thickness here
    
    \center % Center everything on the page
     
    %------------------------------------------------------------------------
    %	HEADING SECTIONS
    %------------------------------------------------------------------------
    
    \textsc{\Large Department of Computing}\\[0.5cm] 
    \textsc{\large Imperial College of Science, Technology and Medicine}\\[0.5cm] 
    
    %------------------------------------------------------------------------
    %	TITLE SECTION
    %------------------------------------------------------------------------
    
    \HRule \\[0.4cm]
    { \huge \bfseries \reporttitle}\\ % Title of your document
    \HRule \\[0.4cm]

    \textit{\reportdescription}
    
    \vspace{2em}

    %------------------------------------------------------------------------
    %	AUTHOR SECTION
    %------------------------------------------------------------------------
    
    \large \emph{Author: Anton Zhitomirskiy}

    \vspace{1em}

    \global\let\newpagegood\newpage
    \global\let\newpage\relax
    
\end{titlepage}

\global\let\newpage\newpagegood

\tableofcontents

\clearpage

\section{Data Augmentation}

\subsection{Input Augmentation}

Input data augmentation is a technique used to increase the size and diversity of your
training set. It does this by applying a series of random but realistic transformations to each data
point during training

By increasing the size of your training data artificially, you're effectively adding more "experiences" for your model, which can improve generalization.

It can be done through \href{https://github.com/aleju/imgaug}{imgaug} or \href{https://pytorch.org/vision/stable/transforms.html}{pytorch.transforms}.

One of the main benefits of data augmentation is its role as a regularizer. It helps prevent overfitting by ensuring that the model encounters a variety of different, yet plausible, examples during training.

\subsection{Transformations}

\begin{itemize}
    \item Random 
    \begin{itemize}
        \item flipping - can simulate the natural orientations of objects or scenes in images
        \item scaling - It helps the model generalize across different sizes of the same object or feature.
        \item rotations - add another layer of complexity by altering the angle of the data points. This is especially useful in tasks like object recognition, where orientation can vary widely.
        \item intensity/contrast variations - different lighting conditions and image qualities
        \item cropping/padding - alter the focus and frame of the data points. This can be helpful for tasks where the subject can be off-center or partially visible
        \item noise - serves as an effective way to improve the model's robustness. It mimics real-world scenarios where data may not always be clean or noise-free
        \item affine transformations - translation, scaling, and shearing. These offer another way to introduce variability into your data, helping your model generalize better
        \item perspective transformations -  alter the viewpoint of the object or scene. This helps in applications like augmented reality, where the perspective can dramatically change the appearance of object
    \end{itemize}
\end{itemize}

\subsection{Anomaly detection}

Pick out the unusualities out of one type of class. 

\subsubsection{Predicit Continuation}

You would train something that is able to regress, to predict the continuation of the function. Then you train some sort of auto encoder that is able from a limited set of inputs to continue this function. If you continue this function and the prediction is too different from your external observation, you reflect that it is an outlier.

\subsubsection{Measure distance in Latent Space}

TODO

\subsubsection{Reconstruct the input}

If you have an auto-encoder, you need to take an input image and reconstruct it. After training, in theory, the network would have a hard time reconstructing something that it has never seen - so it will likely remove it from the image, which means that if you then take the output reconstruction and subtract from the external input which has the anomaly then the error will get highlighted.

\subsubsection{Classify artificial, subtle variations - out of distribution detection}

\subsection{Approaches}

\subsubsection{Unsupervised}

Use auto-encoder reconstruction error and use moving averages, dropout and set time window

\subsubsection{Supervised}

RNNs Learn form a set of yes/nos ina  time series. RNNs can learn from a series of time steps and predict when the anomaly is about to occur.

\subsubsection{Streaming and minibatches}

\end{document}