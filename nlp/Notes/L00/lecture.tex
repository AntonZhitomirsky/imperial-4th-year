\documentclass[11pt]{article}

% some definitions for the title page
\newcommand{\reporttitle}{Intro Lecture and foundations}
\newcommand{\reportdescription}{}

% load some definitions and default packages
%---------------------------------------------------------------------------
%	PACKAGES AND OTHER DOCUMENT CONFIGURATIONS
%---------------------------------------------------------------------------

\usepackage[twoside]{fancyhdr}
\usepackage{csquotes}

\usepackage[a4paper,hmargin=2.0cm,vmargin=1.0cm,includeheadfoot]{geometry}
% \usepackage{natbib} % for bibliography
\usepackage{biblatex}
\usepackage{tabularx,longtable,multirow,subfigure,caption}%hangcaption
\usepackage{fancyhdr} % page layout
\usepackage{url} % URLs
\usepackage[english]{babel}
\usepackage{graphicx}
\usepackage{rotating}
\usepackage{dsfont}
\usepackage{epstopdf} % automatically replace .eps with .pdf in graphics
% \usepackage{backref} % needed for citations
\usepackage{array}
\usepackage{latexsym}
\usepackage[pdftex,hypertexnames=false,colorlinks]{hyperref} % provide links in pdf (had pagebackref)
\usepackage{booktabs}
\usepackage{wrapfig}
\usepackage{caption}  % Required for \captionof
\usepackage{float} % for H option in figures
\usepackage{amssymb}
\usepackage{amsmath}
\usepackage{amsthm}
\usepackage{mathtools} % for 'dcases*' env.
\usepackage[nottoc]{tocbibind}

%%% Default fonts
\renewcommand*{\rmdefault}{bch}
\renewcommand*{\ttdefault}{cmtt}

%%% Default settings (page layout)
\setlength{\parindent}{0em}  % indentation of paragraph
\setlength{\parskip}{.3em}
\setlength{\itemsep}{0.mm}

\setlength{\headheight}{14.5pt}
\pagestyle{fancy}

\fancyfoot[ER,OL]{\thepage}%Page no. in the left on odd pages and on right on even pages

\fancyfoot[OC,EC]{\sffamily }
\renewcommand{\headrulewidth}{0.1pt}
\renewcommand{\footrulewidth}{0.1pt}
\captionsetup{margin=10pt,font=small,labelfont=bf}

% LISTINGS ammendments
\usepackage{listings}
\usepackage{color}

\definecolor{mygreen}{rgb}{0,0.6,0}
\definecolor{mygray}{rgb}{0.5,0.5,0.5}
\definecolor{mymauve}{rgb}{0.58,0,0.82}

\lstset{ 
  postbreak=\mbox{\textcolor{red}{$\hookrightarrow$}\space},
  backgroundcolor=\color{white},   % choose the background color; you must add \usepackage{color} or \usepackage{xcolor}; should come as last argument
  basicstyle=\footnotesize,        % the size of the fonts that are used for the code
  breakatwhitespace=false,         % sets if automatic breaks should only happen at whitespace
  breaklines=true,                 % sets automatic line breaking
  captionpos=b,                    % sets the caption-position to bottom
  commentstyle=\color{mygreen},    % comment style
%   deletekeywords={...},            % if you want to delete keywords from the given language
%   escapeinside={\%*}{*)},          % if you want to add LaTeX within your code
  extendedchars=true,              % lets you use non-ASCII characters; for 8-bits encodings only, does not work with UTF-8
  firstnumber=1,                % start line enumeration with line 1000
  frame=single,	                   % adds a frame around the code
  keepspaces=true,                 % keeps spaces in text, useful for keeping indentation of code (possibly needs columns=flexible)
  columns=fullflexible,
  keywordstyle=\color{blue},       % keyword style
  language=python,                 % the language of the code
  % morekeywords={*,...},            % if you want to add more keywords to the set
  numbers=left,                    % where to put the line-numbers; possible values are (none, left, right)
  numbersep=5pt,                   % how far the line-numbers are from the code
  numberstyle=\tiny\color{mygray}, % the style that is used for the line-numbers
  rulecolor=\color{black},         % if not set, the frame-color may be changed on line-breaks within not-black text (e.g. comments (green here))
  showspaces=false,                % show spaces everywhere adding particular underscores; it overrides 'showstringspaces'
  showstringspaces=false,          % underline spaces within strings only
  showtabs=false,                  % show tabs within strings adding particular underscores
  stepnumber=1,                    % the step between two line-numbers. If it's 1, each line will be numbered
  stringstyle=\color{mymauve},     % string literal style
  tabsize=2,	                   % sets default tabsize to 2 spaces
  title=\lstname% show the filename of files included with \lstinputlisting; also try caption instead of title
}

% Here, you can define your own macros. Some examples are given below.

\newcommand{\R}[0]{\mathds{R}} % real numbers
\newcommand{\Z}[0]{\mathds{Z}} % integers
\newcommand{\N}[0]{\mathds{N}} % natural numbers
\newcommand{\C}[0]{\mathds{C}} % complex numbers
\renewcommand{\vec}[1]{{\boldsymbol{{#1}}}} % vector
\newcommand{\mat}[1]{{\boldsymbol{{#1}}}} % matrix


\usepackage{catchfilebetweentags} % to read input from another file

\bibliography{../bibliography}

\begin{document}

% Include the title page
\begin{titlepage}

    \newcommand{\HRule}{\rule{\linewidth}{0.5mm}} % Defines a new command for the horizontal lines, change thickness here
    
    \center % Center everything on the page
     
    %------------------------------------------------------------------------
    %	HEADING SECTIONS
    %------------------------------------------------------------------------
    
    \textsc{\Large Department of Computing}\\[0.5cm] 
    \textsc{\large Imperial College of Science, Technology and Medicine}\\[0.5cm] 
    
    %------------------------------------------------------------------------
    %	TITLE SECTION
    %------------------------------------------------------------------------
    
    \HRule \\[0.4cm]
    { \huge \bfseries \reporttitle}\\ % Title of your document
    \HRule \\[0.4cm]

    \textit{\reportdescription}
    
    \vspace{2em}

    %------------------------------------------------------------------------
    %	AUTHOR SECTION
    %------------------------------------------------------------------------
    
    \large \emph{Author: Anton Zhitomirskiy}

    \vspace{1em}

    \global\let\newpagegood\newpage
    \global\let\newpage\relax
    
\end{titlepage}

\global\let\newpage\newpagegood

\tableofcontents

\clearpage

\section{Pytorch}

Pytorch will be used for the coursework, recommended is to look at \href{https://pytorch.org/tutorials/beginner/deep_learning_60min_blitz.html}{this} link.

\section{Books}

Recommended NLP Books:

\begin{itemize}
    \item Speech and Language Processing. Dan Jurafsky and James H. Martin~\cite{book-speech-and-language-processing}
    \item A Primer on Neural Network Models for Natural Language Processing.
          Yoav Goldberg~\cite{primer-on-nlp}
    \item Natural Language Processing. Jacob Eisenstein~\cite{git-natural-language-processing}
\end{itemize}

Recommended ML Books:

\begin{itemize}
    \item Artificial Intelligence: a Modern Approach. (2009) Stuart Russell \& Peter Norvig~\cite{AI-modern-approach} with the solutions available at~\cite{AI-modern-approach-slutions}
    \item Machine Learning. (1997) Tom Mitchell~\cite{tom-mitchell-book}
    \item Neural Networks and Deep Learning. Michael A. Nielsen.
    \item Introduction to Deep Learning. Eugene Charniak~\cite{intro-to-dl-eugene-charniak}
    \item Deep Learning by Ian Goodfellow~\cite{Goodfellow-et-al-2016}
\end{itemize}

State of the art NLP:

\begin{itemize}
    \item Papers with code:~\cite{papers-with-code}
    \item  track the progress in Natural Language Processing:~\cite{track-progress-nlp}
\end{itemize}

\section{ML Refresher}

\subsection{Linear activation function}

\ExecuteMetaData[../../../deep-learning/Notes/L03 - Activation and Loss/lecture]{nlp-linear}

\subsection{Non-linear activation functions}

\subsubsection{Sigmoid}

\ExecuteMetaData[../../../deep-learning/Notes/L03 - Activation and Loss/lecture]{nlp-sigmoid}

\subsubsection{ReLU}

\ExecuteMetaData[../../../deep-learning/Notes/L03 - Activation and Loss/lecture]{nlp-relu}

\subsubsection{Tanh}

\ExecuteMetaData[../../../deep-learning/Notes/L03 - Activation and Loss/lecture]{nlp-tanh}

\subsubsection{Softmax}

\begin{figure}[H]
    \centering
    \includegraphics*[width=.3\linewidth]{figures/softmax.png}
\end{figure}

\ExecuteMetaData[../../../deep-learning/Notes/L03 - Activation and Loss/lecture]{nlp-softmax}

\subsection{Loss Functions}

\subsubsection{Mean suqared error}

\ExecuteMetaData[../../../deep-learning/Notes/L03 - Activation and Loss/lecture]{nlp-mse}

\subsubsection{Binary cross-entropy}

\ExecuteMetaData[../../../deep-learning/Notes/L03 - Activation and Loss/lecture]{nlp-bce}

\subsubsection{Categorical cross-entropy}

useful for multi-label classification (predicting one class out of many)

\begin{equation}
    L = - \frac 1 N \sum^N_{i=1}\sum^C_{c=1} y_c^{(i)}\log(\hat{y}_c^{(i)})
\end{equation}

\subsection{Regularization}

\begin{quote}
    any modification we make to a learning algorithm that is intended to reduce its generalization error but not its training error. (See Chapter 7 of~\cite{Goodfellow-et-al-2016})
\end{quote}

\section{Multi-Layer Perceptron Math}

\begin{warning}
    Section not complete. Content of L01.1
\end{warning}

\printbibliography

\end{document}